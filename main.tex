\documentclass[11pt]{article}
\usepackage{a4wide}
\usepackage{times}
\usepackage[french]{babel}
\usepackage[T1]{fontenc} 
\usepackage[utf8]{inputenc}
\usepackage{url}

\usepackage{eurosym}
\usepackage{amssymb}
\usepackage{xcolor}
\newcommand{\mynote}[3][black]{\textcolor{#1}{\fbox{\bfseries\sffamily\scriptsize{#2}}
{\small$\blacktriangleright$\textsf{\emph{#3}}$\blacktriangleleft$}}}
\newcommand{\pem}[1]{} %{\mynote[cyan]{Pierre-Etienne}{#1}}
\newcommand{\ld}[1]{\mynote[magenta]{Laurence}{#1}}
\newcommand{\TODO}[1]{\mynote[red]{TODO}{#1}}
%\newcommand{\TODO}[1]{}
\newcommand{\silent}[1]{}
\newcommand{\gpl}[0]{génie de la programmation et du logiciel}
\newcommand{\eg}[0]{\emph{e.g.},~}
\newcommand{\ie}[0]{\emph{i.e.},~}
\newcommand{\etal}[0]{\emph{et al.}~}
\newcommand{\wrt}[0]{\emph{w.r.t.}~}
\newcommand{\cf}[0]{cf.~}



\title{GdR Génie de la Programmation et du Logiciel\\ Demande de renouvellement pour la période 2021-2025}
\author{Mireille Blay-Fornarino\\
Laboratoire d'Informatique, Signaux et Systèmes de Sophia-Antipolis (I3S)\\ 
UMR7271 - UNS CNRS\\
2000, route des Lucioles - Les Algorithmes - bât. Euclide B\\
06900 Sophia Antipolis - France\\
\\
%%\textbf{DRAFT}
}
\begin{document}
\maketitle



\begin{itemize}
\item Nom de l'unité : GdR Génie de la Programmation et du Logiciel
\item Numéro de l'unité : GDR 3168
\item Institut : INS2I
\item Direction Régionale : Délégation Centre-Est - DR6
\item Code division : 1099
\item Groupe de discipline budgétaire : A1INS2I
\item Date de création : 1er janvier 2008
\item Date de renouvellement : 1er janvier 2021
\item Directeur : Mireille Blay-Fornarino, Professeur à l'Université de Nice Côte d'Azur, Laboratoire I3S
\item Site web : \url{http://gdr-gpl.cnrs.fr/}
\end{itemize}

\section{Motivation et enjeux scientifiques}

Le renouvellement du GDR Génie de la Programmation et du Logiciel (GPL) est
demandé auprès du CNRS pour une nouvelle période de 5 ans allant de janvier
2021 à décembre 2025.
L'objectif du GDR GPL est d'animer la communauté scientifique dans
le domaine du Génie Logiciel et de la Programmation, et également d'être un outil de prospective scientifique dans ce domaine par les retours de notre communauté et ses collaborations avec d'autres communautés.



%Jeunes chercheurs


%\section{Motivation}

Le Génie de la Programmation et du Logiciel est au c{\oe}ur de l'activité
informatique. Les concepts, méthodes et les outils de conception et de
validation de logiciels constituent les éléments manipulés par les
informaticiens pour maîtriser et automatiser les problèmes qui leur sont
soumis. 
Plusieurs grandes familles thématiques ont été définies lors des actions de prospective que nous avons menées en 2010 et 2014, et qui avaient conduit à plusieurs Groupes de Travail (GT). \\
Fin 2019, nous avons lancé une nouvelle action dans ce sens (voir texte en annexe). %Nous travaillons encore sur les retours, mais nous avons déjà identifié, les grandes thématiques suivantes, à savoir :\\
Les grandes thématiques du domaine sur lesquelles les chercheurs du
domaine travaillent sont, au regard des 21 défis reçus en décembre 2019~: 

\begin{itemize}
  \renewcommand{\labelitemi}{$\bullet$}
\item l'ingénierie dirigée par les modèles qui exploite une continuité entre  modélisation, conception, programmation et développement pour systématiser la production et la maintenance de logiciels plus sûrs et plus agiles, elle inclut~:
\begin{itemize}
\item l'ingénierie des exigences et de leurs interactions avec les artefacts logiciels pour assurer la traçabilité logicielle, y compris dans des objectifs d'argumentation et de validation de la justesse des systèmes produits, 
\item la séparation et la composition de préoccupations, par exemple la sécurité, pour maîtriser la conception, le développement et l'évolution de grands systèmes, 
\item la modélisation et la gestion de la variabilité logicielle à des fins de production, de configuration et de maintenance de lignes de produits logiciels;
\end{itemize}

\item la conception et l'outillage de langages plus sûrs et plus expressifs,  elle inclut~: 
\begin{itemize}
\item les approches de compilation pour des architectures spécifiques et notamment parallèles, y compris ces dernières années pour l'exploitation des GPU et  des ordinateurs quantiques,
\item la conception et l'utilisation de langages métiers et de langages spécifiques à des domaines,
\item la certification des chaînes de compilation,
\item de nouvelles approches de la réparation de "bugs";
\end{itemize}
%et s'appuie sur des techniques de  transformation de modèles,

\item le développement de techniques de vérification et de validation des systèmes, elle inclut~:
\begin{itemize}
\item la vérification et la validation à partir  de spécifications ou de code : analyse statique, génération de tests et raffinements prouvés,
\item la prise en compte de la testabilité logicielle de l'élaboration des systèmes logiciels à leur adaptation en production, y compris dans des contextes d'intégration d'intelligence artificielle, 
\item l'adaptation de ces travaux aux systèmes émergents tels que les systèmes cyber-physiques, les systèmes intégrant de l'IA, les véhicules autonomes;
\end{itemize}
\item la prise en compte dans l'ensemble du cycle de vie du logiciel des propriétés telles que la  sécurité, la sobriété écologique, la résilience, elle inclut~:
\begin{itemize}
\item la recherche sur de nouvelles architectures logicielles et méthodes de développements pour faire face aux nouvelles contraintes : taille des systèmes, mondialisation du développement, protection des données, gestion des ressources, déconnexion, ...
\item l'adaptation des approches de développement pour intégrer la maintenance et l'adaptation dynamique des systèmes aux changements de contexte, y compris par apprentissage automatique, en particulier pour les systèmes cyber-physiques;
\end{itemize}
\item l'étude des pratiques de développement y compris de maintenance logicielle, notamment par des études empiriques et la mise au point d'abstractions adaptées;
\item  l'exploitation des travaux sur l'IA dans un contexte de \gpl et inversement, dont les travaux cités précédemment.

\end{itemize}

Sans vouloir être exhaustif, ces grandes thématiques regroupent une grande partie des
thèmes de la programmation et du génie logiciel abordés par les équipes
françaises actuellement. 
\noindent
Avec l'omniprésence de l'informatique dans notre vie que ce soit en termes
d'informatique embarquée, d'intelligence ambiante, d'extension du web au niveau
de la planète, d'intégration dans les objets du quotidien, ou encore avec le
développement de grandes infrastructures de calcul ou de traitement de grandes
masses de données, de nouvelles questions de recherche sont posées.
De nouveaux paradigmes, de nouveaux langages, de nouvelles approches de
modélisation, de vérification, de tests et de nouveaux outils dans le domaine
de la programmation et du logiciel devraient voir le jour dans les 5 à 10 ans à
venir, que ce soit pour faciliter la vie des concepteurs de logiciels, pour
modéliser et fiabiliser les logiciels ou encore pour devancer l'évolution
technologique, mais également pour prendre en compte de nouveaux enjeux de
société tels que le développement durable, les économies d'énergie ou la maîtrise des systèmes intégrant de l'intelligence artificielle.

% Défis
% -----
% * Des exigences à la réalisation des logiciels
%   - Modéliser et analyser des Systèmes de Systèmes (sécurité, développement et évolution dyamique)
%   - Gérer et modéliser la variabilité à grande échelle (lignes de produits)
%   - Composer et coordonner des langage de modélisation spécifiques (DSL)

% * De la compilation à l'exécution des logiciels
%   * Maitriser la difficulté de programmation des architectures multi-coeurs
%   Assurer la portabilité des codes et des performances
%   - Optimiser les mouvements de données et savoir prédire les temps
%     d'exécution en présence de parallèlisme
%   - Assurer la validité des transformations effectuées par les compilateurs
%     [a deplacer dans la section vérification]
%   - Etre capable de reproduire les expériences et de vérifier les résultats,
%     en particulier dans le domaine de l'optimisation du logiciel
%   * Inventer une nouvelle façon de considérer les pilotes de périphériques en
%     s'inspirant du domaine génomique pour faciliter leurs adaptations
%   * Optimiser la consommation d'énergie des logiciels s'exécutant sur des
%     architectures multi-coeurs [a factoriser avec la section précédente]

% * Méthodes et outils de validation et de vérification
%   - Inventer des méthodes de validatidation et de vérification plus
%     facilement paramétrables ou automatisable pour s'adapter à l'évolution des
%     systèmes
%   - Permettre au méthodes basées sur les tests de passer plusfacilement à l'échelle
%   - Concevoir des logiciels capable de s'adapter aux défaillances en corrigeant éventuellement les fautes




%(1) la modélisation des logiciels prenant en compte la variabilité à grande échelle y compris pour les adapter dynamiquement aux évolutions de contexte;
%(2) l'intégration dans les langages de programmation et dans les processus de développements de formalisations permettant de construire des systèmes plus sûrs, exploitant au mieux les infrastructures modernes, de renforcer notre confiance dans ces systèmes, et dans le même temps, de masquer la complexité de ces vérifications par des abstractions adaptées; 
%(3)l'optimisation de la consommation d'énergie des logiciels et de leur résilience, (4) l'exploitation des masses de projets logiciels pour empiriquement apprendre et démontrer les avantages et les biais des différentes approches de développement, (5) l'exploitation des travaux sur l'IA dans un contexte de \gpl et inversement, y compris pour étendre les approches de tests, vérifications, validation au monde du Machine Learning. 
%la modélisation des systèmes de systèmes, 
%la maîtrise de la programmation des architectures multi-c{\oe}urs,

%l'étude des pilotes de périphériques en s'inspirant du domaine génomique,
%et, finalement, la place des méthodes et outils de vérification et de validation. 
%


%
%\TODO{ D'autres thèmes non identifiés en tant que tels suscitent des débats et donnent également de nouveaux enjeux pour les travaux de recherche futurs dans le domaine. Il s'agit du continuum dans le développement, des approches agiles, de l'association entre connaissance et développement, du dialogue avec l'utilisateur pendant les phases de développement et de l'apprentissage par les usages, le développement et l'exécution dans des environnements ouverts, la complexité et le passage à l'échelle dans l'utilisation des approches formelles, de l'utilisation de vues dynamiques pour manipuler ou remodulariser des  objets existants, et, finalement la contractualisation et la responsabilité face aux utilisateurs du logiciel.}

Dans ce document, nous rappelons les objectifs du GDR GPL et explicitons comment nous souhaitons fonctionner pendant le nouveau quinquennal. 


\section{Objectifs}
La proposition de renouvellement du GDR Génie de la Programmation et du
Logiciel (GPL) vise des objectifs (1)  d'animation de la communauté scientifique
avec la proposition d'un cadre structurant pour un ensemble de groupes de
travail qui réunissent les équipes françaises actives dans le domaine du GPL, (2) de veille et prospective scientifique dans ce cadre par des actions dédiées, (3) de diffusion des connaissances du domaine du GPL y compris en articulation avec d'autres domaines.

%Onze groupes et une action transversale sont présentés dans ce document, mais le GDR reste ouvert à de nouvelles propositions. Cinq groupes existants (Compilation, LTP, MFDL, MTV2, RIMEL) et les actions IDM et AFSEC dans le quadriennal précédent ont demandé leur reconduction. Quatre nouveaux groupes sont créés (ESE, GLACE, IE, SdS) et une nouvelle action transversale est créée (ALROB). Cette répartition couvre l'ensemble des travaux de la communauté française dans le domaine du Génie de la Programmation et du Logiciel.

Le GDR se veut être un groupement où la réflexion et les échanges sont permanents.
Dans la continuité du précédent quinquennal, le GDR apportera  son soutien aux groupes de travail pour leurs activités d'animation (organisation de journées, généralement sur base annuelle). 
Il soutiendra l'organisation de colloques et ateliers associés au GDR lorsque ceux-ci seront organisés par les groupes de travail.
Le GDR aura également pour mission d'organiser les journées nationales, afin de permettre la rencontre des diverses communautés qui constituent le GDR. Certaines de ces activités d'animation se feront en lien avec les chercheurs d'autres domaines, en particulier par des actions transverses aux GDR.
Le GDR encouragera la mobilité de chercheurs en offrant un soutien financier. Il apportera son soutien à l'École des Jeunes Chercheur·e·s en Programmation sous la forme d'interventions dans les différents cours de l'école, mais également par un soutien financier. 

%En sus de ces actions, nous proposons de nouvelles actions pour la période 2021-2025.
Nous rappelons ci-dessous le fonctionnement de l'animation de la communauté, nous développons les actions vers les jeunes chercheur·e·s et nous  proposons de nouvelles actions de diffusion et prospective scientifique intégrant un  développement vers l'international et vers le monde industriel.
L'organisation nécessaire à soutenir ces nouvelles approches est présentée au paragraphe suivant (\cf \ref{organisation}).
%\section{Propositions pour la période 2012-2015}\label{actions}

\subsection{Animation de la communauté}
L'animation se traduit par
    l'organisation de journées nationales, 
    des réunions de ses groupes de travail,
    le soutien à des actions spécifiques, 
    mais également par l'utilisation d'outils de communication
    et de partage tels que la liste de diffusion et le site web, et des actions ponctuelles telles qu'une réflexion de prospection sur les défis scientifiques du domaine.
    
\subsubsection{Journées nationales}

Les journées nationales constituent un temps fort dans la vie du
GDR GPL réunissant plus de 120 personnes chaque année. Nous comptons bien sûr reconduire ces journées, toujours en les co-localisant à des conférences ou journées nationales du domaine.
Elles permettent aux chercheurs du domaine de se retrouver et d'échanger, et
également de prendre connaissance de domaines connexes. 
Pour garantir la qualité de ces journées, nous continuerons à impliquer les groupes de travail dans la constitution du programme scientifique. En particulier, nous envisageons lors de ces journées de donner une large place à des groupes de travail transverses au GPL et à d'autres domaines comme les recherches à la frontière avec l'IA, l'IOT ou le développement durable qui soulèvent des problématiques abordées par plusieurs communautés du GDR. 

La prochaine édition des journées aura lieu à Vannes du 16 au 19 juin 2020. Elle devrait être co-localisée avec les ateliers AFADL et les conférences nationales CIEL et CAL.

%, ainsi qu'avec l'Ecole Jeunes Chercheurs en Programmation. Cette co-localisation avec l'Ecole Jeunes Chercheurs devrait permettre une plus grande participation des jeunes doctorants. Nos collègues de Nancy sont d'ores et déjà candidats pour l'édition 2013, ce qui montre la vitalité de ces journées et nous encourage à poursuivre l'effort entrepris pour que ces journées réunissent des événements scientifiques de qualité.

\subsubsection{Animation scientifique par les groupes de travail et actions}

Les groupes de travail sont encouragés à organiser des réunions de travail
tout au long de l'année. 

Des groupes de travail au c\oe{}ur du génie de la programmation et du logiciel mais avec des contours plus interdisciplinaires que pendant le dernier quinquennal pourraient se constituer et développer de nouvelles formes d'animation (\eg réunions de travail thématiques, contributions à des actions internationales). 

Des actions ponctuelles seront également encouragées à visée de nouveaux groupes de travail, d'étude prospective (\eg rédaction du manifeste\footnote{\url{http://gdr-gpl.cnrs.fr/node/369}} lors du dernier quinquennal), de médiation, ... 
Nous travaillerons sur ces points lors de la journée du 16 mars à Evry. 

De plus nous nous appuierons sur les membres du comité scientifique pour proposer des animations spécifiques telles que une sensibilisation à la préservation des logiciels à des fins de reproductibilité ou des journées inter-GDRs. 

Le GDR GPL apportera son soutien financier pour faciliter l'intendance de ces journées et actions.



\subsubsection{Mobilité}
Le GdR Génie de la Programmation et du Logiciel a mis en place depuis 2010 une action d'incitation à la collaboration et à la mobilité entre équipes. Ces mobilités ont pour objectif de favoriser des travaux communs
entre chercheurs. Des bourses d'un montant forfaitaire de 500 à 1000\euro~ ont été proposées. Elles sont destinées à couvrir des frais de déplacement et de logement d'un doctorant, d'un chercheur ou enseignant-chercheur pour un séjour de courte durée dans une autre équipe, géographiquement distante (située dans un autre département). Ces mobilités ont permis la publication d'articles, mais aussi l'amorce de projets collaboratifs. Nous pensons reconduire cette action dans le prochain quinquennal.

\subsection{Actions vers les jeunes chercheur·e·s}

Le GDR GPL a à c{\oe}ur d'inclure l'ensemble des jeunes doctorants dans la
communauté. Dans ce sens plusieurs actions sont déjà organisées que nous reconduirons et nous en proposons de nouvelles.

\begin{itemize}
\item L'École des Jeunes Chercheur·e·s en Programmation (EJCP), créée en 1991, réunit
  chaque année une quarantaine de jeunes doctorants dans les thématiques du GDR
  GPL. Le GDR est l'un des sponsors principaux de cette manifestation.
  L'aide du GDR GPL a permis de maintenir des tarifs d'inscription très bas, incluant
  les frais d'hébergement, pour les jeunes chercheur·e·s participants. 
  
\item Le GDR GPL soutiendra également d'autres écoles d'été, comme cela a été
  fait sur le précédent quinquennal sur recommandation des groupes de travail
  qui y seront fortement impliqués. 
  
\item Le GDR GPL consacrera une partie de son budget pour permettre aux doctorants de participer à des conférences
nationales et internationales dans le domaine du Génie Logiciel et de la
Programmation. Ces soutiens se feront à la demande d'un groupe de travail
lorsque celui-ci sera impliqué dans l'organisation de cette manifestation.



\item Depuis quelques années les inscriptions aux journées nationales sont
  offertes ou proposées à un tarif très bas aux jeunes chercheur·e·s
  présentant leur travail. Nous souhaitons continuer dans ce sens de façon à
  accueillir les jeunes doctorants dans la communauté et à leur permettre
  rapidement de se construire un réseau professionnel.
  
  \item Pendant les EJCP, une courte session sera consacrée chaque année, à présenter le concours CNRS comme un objectif qui se prépare dès le début de la thèse et à leur donner des éléments pour s’y projeter.

\item Pendant les journées nationales, pour les jeunes docteurs et doctorants en dernière année, un atelier de préparation aux concours de chercheurs et enseignants-chercheurs leur sera proposé.

  
\item Un prix de thèse dans le domaine du Génie
  Logiciel et de la Programmation a été créé en 2013. 
  Nous continuerons à organiser ce prix permettant de reconnaître les travaux les plus
  marquants des jeunes docteurs du domaine. Un prix spécial sera créé pour les thèses réalisées dans un contexte industriel.
  %En effet, les prix de thèse existants récompensent peu les thèses de notre domaine. 
  Cette action continuera à s'appuyer fortement sur le comité scientifique du GDR.
  
  \item Enfin nous envisageons d'impliquer davantage les "jeunes chercheur·e·s" de notre communauté par exemple en encourageant des co-directions de groupes de travail.
 %Nos travaux sont des travaux au c\oe ur de la discipline informatique et non aux interfaces, donc moins visibles. 
 \end{itemize}



%\subsection{Partage au sein de la communauté GDR GPL}



%, mais également d'être plus présent en nombre lors des grands événements internationaux. 

%\subsubsection{Cartographie des laboratoires}
%\pem{à développer}

%ou dans les comités scientifiques des appels à projets. 
%
%, le soutien dans la réponse aux appels à projets de type ANR \\
%De même, les réponses aux appels à projets demandent une connaissance à la fois des techniques d'écriture des projets, mais également du système de sélection. Nous proposerons, également sur la base du volontariat, une relecture commentée de ces réponses.

%\subsubsection{Répertoire de documents scientifiques et d'outils de recherche}
%
%Notre communauté produit de très nombreux documents scientifiques (articles, thèses, rapports) accessibles au travers de moteurs de recherche. 
%L'idée est d'organiser leur accès par thématique en listant par exemple les articles de références ou encore la liste des thèses soutenues, de façon à aider les personnes débutant sur un sujet.
%Ce travail pourra être mené par les groupes de travail.
%\\
%Il en va de même pour les outils de recherche. Avoir un répertoire contenant un descriptif et des retours d'évaluation de ces outils 
%rs du précédent quadriennal
%permettrait un gain de temps lorsque l'on doit faire choisir, que ce soit dans le montage d'expérimentation ou encore pour la validation de résultats.

%\subsection{Enseignement du Génie de la Programmation et du Logiciel}

%Nombre d'entre nous sommes enseignants et chercheurs dans des universités ou des écoles d'ingénieurs. Par notre enseignement, nous diffusons les connaissances de notre domaine de recherche vers de jeunes étudiants en vue de les former aux nouvelles avancées scientifiques et techniques.

%\subsubsection{Réflexion sur un cursus GPL}

%Construire un programme qui couvre notre domaine est un exercice que nous faisons régulièrement lors de l'écriture des habilitations de Licence ou Master. L'objectif de cette action est de réfléchir à ce que doit contenir un cursus Génie de la Programmation et du Logiciel. L'enjeu est de proposer un cursus de qualité aux futurs diplômés, mais également de former des jeunes diplômés ayant des connaissances solides dans le domaine et qui trouveront à s'insérer facilement dans le monde socio-professionnel. 
%\\
%Pour parvenir à convaincre la communauté universitaire de la nécessité de définir un cursus commun et le pérenniser dans le cadre des formations orientées Génie Logiciel, ce travail devra être mené conjointement avec l'association SPECIF.

%\subsubsection{Répertoire pédagogique} 

%Les formes d'enseignement de notre domaine sont nombreuses, qu'il s'agisse d'outils, de langages, d'exercices, de projets ou d'initiatives particulières. 
%L'objectif de cette action est de répertorier les formes les plus utilisées, de les partager en les commentant en vue de créer une communauté active autour de l'enseignement du domaine.


\subsection{Actions de prospective, développement vers l'international  et vers le monde de l'entreprise}


\subsubsection{Prospectives scientifiques}
En 2010, 2014, 2019-20 des actions de prospective scientifique dédiées à
l'identification des nouveaux défis pour le GDR Génie de la Programmation et du
Logiciel ont été menées. Les documents 2010\footnote{\url{http://gdr-gpl.cnrs.fr/sites/default/files/documentsGPL/Defis2025/DefisGPL2010.pdf}} et 2014 \footnote{\url{http://gdr-gpl.cnrs.fr/sites/default/files/documentsGPL/Defis2025/RapportConjontureGdrGPL2014.pdf}} sont disponibles sur le site web du GDR et ont été publiés dans TSI.
Le document 2020 est en cours de rédaction et sera
diffusé en archive ouverte; il comprendra une première partie sous la forme d'un rapport de	conjoncture et une deuxième partie présentant les défis identifiés par la communauté du GDR GPL. \\

Mener ce type d'action régulièrement est
nécessaire pour identifier les nouveaux défis ainsi que les avancées du
domaine. 
Afin de répondre aux nouvelles missions des GDR, nous demanderons aux groupes de travail de faire un bilan prospectif tous les 2 ans et proposerons de mener des actions de prospectives globales à l'ensemble de la communauté en 2024-2025.


\subsubsection{Développement des relations industrielles, nationales, et internationales}
Notre communauté contribue très largement à proposer des solutions bien fondées et pratiques permettant la production et la maintenance de logiciels. 
Pour cela, les chercheurs du GDR travaillent sur des abstractions logicielles et les fondements scientifiques
associés, qui interviennent à toutes les étapes du cycle de vie du logiciel, de sa conception à son exécution, et prennent forme dans des théories, techniques et
méthodes de modélisation, de validation, de vérification, dans les langages de programmation, dans les
plates-formes d’exécution ou encore lors de la mise en place de tests.(\cf rapport de conjecture en 2014).


Présentées soit d'un point de vue applicatif, soit comme support aux autres sciences, soit comme un état de fait, ces recherches en GPL apparaissent comme anecdotiques dans les appels ANR et sont trop souvent confondues avec du support technologique dans les entreprises. Dans le même temps, la validation des recherches dans notre domaine exige de plus en plus des expérimentations à très grande échelle.
Pour tenter d'apporter des réponses à ces difficultés de présentation et de mise en oeuvre, nous avons ajouté au comité de direction du GDR un responsable des interactions avec les entreprises et instituts et un responsable des relations internationales, qui aideront, par différentes actions, notre communauté à mieux se positionner.

Les prochaines actions envisagées sont une prise de contact avec différents IRT (IRTB, SystemX, Saint-Exupéry), et un positionnement de nos travaux de recherche relativement à l'association "Informatics Europe"\footnote{\url{https://www.informatics-europe.org/}}.
Par des interactions avec ces différents partenaires sociaux-économiques naturels du monde du logiciel, notre ambition est double~: faciliter un positionnement de notre communauté et diffuser les résultats de nos recherches théoriques et appliquées.\\
%Plusieurs groupes de travail ont développé des relations  suivies avec des équipes internationales \TODO {(\eg Groupe Compilation)}. Nous souhaitons renforcer ces interactions en fonction des groupes de travail en les rendant plus visibles (\eg interactions avec des organismes de certification, animations à l'international). \\
Nous envisageons également de proposer un prix spécial pour les thèses réalisées dans un cadre industriel, prix qui n'existe pas aujourd'hui dans notre domaine.

%Au cours du prochain quadriennal nous envisageons de mettre en place un certain nombre d'outils et de manifestations permettant d'établir des échanges structurés avec le milieu industriel. Nous pensons en particulier effectuer une cartographie des laboratoires, permettant de recenser les compétences, les collaborations et les formations existantes. Nous organiserons également des séminaires permettant d'échanger afin de mieux se connaître et de mieux cerner les besoins et les attentes aussi bien du milieu industriel que du milieu académique.



%\subsubsection{Développement à l'international}
%\TODO{Je manque d'éléments là dessus...}
%Plusieurs groupes de travail ont développé des relations  suivies avec des équipes internationales.
%On peut noter par exemple l'action IDM qui a organisé une master class à Oslo avec le SINTEF en 2010. 
%\TODO{Nous encouragerons le développement de ces relations et proposerons à différents groupes de travail de monter un GDRI sur le prochain quadriennal. }

%\subsubsection{Soutien dans l'écriture d'articles de recherche sur les grandes conférences et sur les réponses aux appels à projets}
%\TODO{L'un des constats du précédent quinquennal est qu'il est nécessaire que notre communauté soit plus présente dans les programmes des grandes conférences internationales,  dans les grands journaux internationaux et également dans les actions internationales sur le génie de la programmation et du logiciel. Cela passe par différentes actions qui sont le mentorat pour l'écriture d'article, mais également par le renforcement de la présence de notre communauté dans les comités de programmes des conférences internationales, dans les comités d'édition des journaux. De grandes conférences internationales telles que la conférence ICSE proposent un système de relecture d'articles par un mentor. L'idée est de proposer à ceux qui le souhaitent ce type de service, sur la base du volontariat.}



%Nous allons L'une des premières actions sera l'organisation d'une rencontre Industries du logiciel - GDR dans l'année 2011-2012 sur la ville de Toulouse. Cette journée sera un prélude à la création de ce club de partenaires. 
%Les entreprises adhérentes à ce club de partenaires pourront alors bénéficier d'une cartographie des laboratoires du domaine et avoir des relations privilégiées  avec les différents groupes de travail. Déjà, le groupe NEPTUNE, qui regroupe les industriels utilisateurs des technologies issues de l'IDM (Ingénierie Dirigée par les Modèles), tels que C-S, Astrium ou Airbus, organise régulièrement des journées rencontres industries-académiques et participent aux activités du GDR GPL. 
% Nous comptons également nous appuyer sur les pôles de compétitivité tels que Systematic, Minalogic, AeroSpace Valey ou encore SCS pour renforcer les liens avec le tissu industriel.



\section{Renouvellement de l'organisation du GDR GPL}\label{organisation}

La structuration du GDR GPL passe essentiellement 
par le comité de direction, les groupes de travail, les actions transversales et le comité scientifique.

%L'animation se traduit 
%    par des réunions de ces groupes et actions,  
%    l'organisation de journées nationales, 
%    le soutien à des actions spécifiques, 
%    mais également par l'utilisation d'outils de communication
%    et de partage tels que la liste de diffusion et le site web, et des actions ponctuelles telles qu'une réflexion de prospection sur les défis scientifiques du domaine. 

%\pem{mettre à jour}
%Plus de 80 équipes se répartissent sur les 10 groupes ou actions,
%soit environ 350 enseignants-chercheurs, 50 postdocs et ingénieurs et 300
%doctorants.

%Les principales actions de l'animation du GDR GPL sont issues des membres du
%Comité de Direction et des responsables des groupes de travail. 
Pour préparer ce dossier de renouvellement nous avons lancé un appel à défis, identifié avec un large comité différents points de convergence entre les défis, proposé des regroupements de défis, conseillé les porteurs de défis dans la construction de groupes de travail. L'étape suivante prévue le 16 mars 2020 est de présenter les défis et de former des ateliers pour élaborer de nouveaux groupes de travail, sans bien évidemment interdire aux groupes dynamiques d'être reconduits. Nous désirons en particulier lors de cette journée motiver des co-directions de groupes de travail avec de jeunes chercheur·e·s.

%\TODO{Pour préparer ce dossier de renouvellement nous avons échangé avec les différents responsables de groupe de travail et les membres du comité de direction. Pour permettre au GDR de continuer à être dynamique et innovant nous avons encouragé la création de nouveau groupes et c'est tout naturellement que plusieurs responsables de groupes ou membre de l'équipe de direction ont proposé de passer le relai à d'autres collègues.}


\subsection{Comité de direction}

La direction du GDR GPL est organisée autour d'un comité de direction. 
Le rôle du comité de direction sera intensifié pour répondre aux nouvelles demandes de l'INS2I en matière de prospectives, favoriser la diffusion des travaux de notre communauté, et faciliter et accroître les interactions scientifiques avec les partenaires nationaux et internationaux. \\
Le comité de direction se réunira au moins quatre fois par an par téléphone pour faire le point sur les actions menées par le GDR, discuter du budget et de l'organisation des
principaux événements (journées nationales, actions de prospective, actions
spécifiques, école jeunes chercheur·e·s.). Le comité de direction se réunira
également chaque année à l'occasion des journées nationales. Outre ces réunions, la plupart des décisions d'attribution d'un
soutien financier du GDR seront discutées par email.

Nous donnons la
composition du comité, sur le quinquennat précédent (2016--2020) et le quinquennat à venir
(2021--2025), puis explicitons les responsabilités au sein du comité.

\begin{small}
\begin{center}
\begin{tabular}{c|c}
  \textit{2016--2020} & \textit{2021--2025}\\
  \\

\hspace*{0.2cm}
\begin{minipage}[t]{.5\textwidth}
  \textbf{Un directeur du GDR :}
  \begin{quote}
    \textbf{Pierre-Etienne Moreau},
  Professeur à l'Université de Lorraine, École des
        Mines de Nancy, laboratoire LORIA
  \end{quote}
  \textbf{Responsables du pôle \textit{Langages et Vérification} :}
  \begin{quote}
    \textbf{Yamine Ait Ameur}, Professeur à l'ENSEEIHT, laboratoire IRIT

%%% PEM: Ce n'est pas Samovar ?
    \textbf{Catherine Dubois}, Professeur à l'École Nationale Supérieure d'Informatique pour l'Industrie et l'Entreprise, laboratoire Cedric
  \end{quote}

  \textbf{Responsables du pôle \textit{Développement de Logiciel} :}
  \begin{quote}
    \textbf{Mireille Blay-Fornarino}, Professeur à l'Université de Nice-Sophia-Antipolis et au laboratoire I3S

    \textbf{Xavier Blanc}, Professeur à l'Université de Bordeaux, laboratoire LABRI
\end{quote}

\textbf{Responsable de la communication :}
\begin{quote}
  \textbf{Yves Ledru}, Professeur à l'Université de Grenoble et au LIG
\end{quote}

\textbf{Responsable de l'Ecole des jeunes chercheur·e·s. en Programmation :}
\begin{quote}
  \textbf{Jean-Christophe Filliatre}, Directeur de Recherche au CNRS,  LRI
\end{quote}
\end{minipage}
&
\begin{minipage}[t]{.5\textwidth}
\textbf{Un directeur du GDR :}
  \begin{quote}
    \textbf{Mireille Blay-Fornarino}, Professeur à l'Université Nice Côte d'Azur, laboratoire I3S
  \end{quote}

  \textbf{Responsable du pôle \textit{Langages et Vérification} :}
  \begin{quote}
     \textbf{Catherine Dubois}, Professeur à l'École Nationale Supérieure d'Informatique pour l'Industrie et l'Entreprise, laboratoire SAMOVAR
  \end{quote}

  \textbf{Responsables du pôle \textit{Développement de Logiciel} :}
    \begin{quote}
      \textbf{Jean-Michel Bruel}, Professeur à l'Université de Toulouse,
laboratoire IRIT
     \end{quote}

  \textbf{Responsable des relations internationales} :
    \begin{quote}
      \textbf{Jean-Marc Jezequel}, Professeur à l'Université Rennes 1, laboratoire IRISA
     \end{quote}
  
  \textbf{Responsable des relations avec les entreprises et instituts :}
   \begin{quote}
      \textbf{Xavier Blanc}, Professeur à l'Université de Bordeaux,  LABRI
    \end{quote}

    \textbf{Responsable de la communication :}
    \begin{quote}
        \textbf{Yves Ledru}, Professeur à l'Université de Grenoble, LIG
    \end{quote}

\textbf{Responsable de l'Ecole des Jeunes Chercheur·e·s. en Programmation :}
\begin{quote}
  \textbf{Laure Gonnord}, maîtresse de conférences (HDR) à Université Claude Bernard Lyon 1, LIP
  \end{quote}
  
\textbf{Président du comité scientifique :}
\begin{quote}
   \textbf{Yamine Ait Ameur}, Professeur à l'ENSEEIHT, laboratoire IRIT
  \end{quote}
\end{minipage}

\end{tabular}
\end{center}

\end{small}
\medskip

\noindent


\textbf{L'organisation en pôles} assure une représentation des deux grands axes du GDR. Jean-Michel Bruel et Catherine Dubois connaissent tous les deux très bien nos communautés et seront des vecteurs naturels d'interactions entre les équipes des 2 pôles. \\
L'intégration dans le comité de direction d'un responsable des \textbf{relations internationales}, Jean-Marc Jezequel, devrait permettre d'amplifier les actions de veille et prospective avec notre communauté scientifique internationale et nationale et également vis à vis de la direction de l'INS2I. \\
%Les recherches en \gpl~tirent parfois leurs objets d'études des pratiques en entreprises et impactent également ces mêmes pratiques, pouvant conduire à une confusion entre recherche et ingénierie. 
Pour favoriser une meilleure compréhension et diffusion de la recherche en GPL en particulier face aux nouveaux défis sociétaux en prise avec le logiciel, Xavier Blanc a pris la responsabilité des interactions scientifiques avec les \textbf{entreprises, les IRT, l'ANR, ...}.\\
Yves Ledru continuera à assurer la \textbf{communication } du GDR au travers du site web, des listes de diffusion, ... Nous envisageons de mettre en place d'autres outils pendant ce quinquennat, pour faciliter une contribution de l'ensemble de la communauté à la communication et à répondre aux attentes de la direction de l'INS2I en matière de cartographie des équipes en GPL sur le territoire.\\
La \textbf{responsabilité de l'école des Jeunes Chercheur·e·s.} sera portée par Laure Gonnord, avec l'objectif de rapprocher chaque année, les formations données au sein de l'école des recherches des équipes accueillantes et promouvoir auprès des jeunes les concours CNRS. \\
Afin d'accroître les interactions avec le comité scientifique, nous avons demandé à Yamine Ait Ameur de prendre la \textbf{présidence du comité scientifique} et d'en même temps participer au comité de direction. Nous souhaitons en particulier nous appuyer sur le comité scientifique pour répondre aux demandes de l'INS2I en experts, envisager des réorientations de nos actions, être un vecteur d'animation envers l'ensemble de notre communauté et parfois également à la frontière avec d'autres communautés et, comme par le passé, aider au prix de thèse du GDR.



Forte d'une équipe créative et active, la direction du GDR GPL souhaite à la fois être force de propositions vis à vis de notre communauté et des décideurs, et un catalyseur des mouvements nationaux et internationaux, académiques ou industriels qui émergent ces dernières années pour faire face à un accroissement exceptionnel de la place du logiciel dans nos sociétés.



\subsection{Groupes de travail et actions}
A ce stade, nous n'avons pas encore constitué les groupes de travail.
Nous explicitons ici la méthode que nous proposons de suivre pour construire nos nouveaux groupes de travail.

%\subsubsection{Vision prospective des défis portés par la communauté}
Catherine Dubois, Pierre-Etienne Moreau et moi-même avons lancé fin décembre un appel à défis à l'ensemble de notre communauté et également via la SIF. 
Nous avons constitué un  comité d'études des défis composés de 33 chercheurs avec un objectif de couverture nationale, thématique, paritaire et d'intégration des jeunes chercheur·e·s CNRS recruté·e·s dans les thématiques de notre GDR ces dernières années (voir annexe pour la constitution du comité). 
Nous avons reçu 21 propositions de défis.
Au moins 3 relecteurs issus du comité des défis ont évalué ces défis selon les axes suivants : pertinence du défi, recouvrement avec d'autres défis\footnote{affectation permettant de couvrir plusieurs défis potentiellement liés, et tous les défis étaient accessibles à tous les relecteurs}, perspectives relativement à un groupe de travail. Lors d'une réunion téléphonique du comité de sélection, nous avons dégagé 12 grandes familles de défis scientifiques auxquels notre communauté s'intéressent. Pour chacun des défis, nous avons donné un avis et des conseils pour la constitution d'un groupe de travail et des actions d'animation.\\
Les porteurs des défis sont à présent invités à collaborer pour les rédiger afin de les diffuser largement. La journée du 16 mars devra donc leur permettre de mettre au point ces rapprochements si cela est nécessaire et de travailler à l'élaboration des groupes de travail du GDR.

%\subsubsection{Groupes de travail}
En regard des défis proposés, nous envisageons déjà deux sortes de groupes de travail : \\
- des groupes centrés \gpl~qui regroupent plusieurs équipes (\eg un GT Compilation); ils sont un espace d'échanges scientifiques entre spécialistes avec des animations adaptées; \\
- des groupes interdisciplinaires qui abordent selon différents angles du \gpl~une thématique connexe au GL (\eg IA, sobriété écologique); ils nécessitent des animations transverses dirigées par des problématiques du \gpl~(\eg vérification et algorithmes de ML; adaptation dynamique au contexte) et peuvent répondre à des demandes d'autres GDRs par exemple.

D'autre part, des actions spécifiques seront encouragées par exemple pour préparer un nouveau groupe de travail. 

Les groupes de travail seront sollicités pour organiser des journées de travail et participer à l'animation des journées nationales. 
%Ils contribueront également à l'organisation de conférences nationales ou internationales.
 Tous les 2 ans, nous  demanderons aux groupes de travail de faire bref compte-rendu des actions menées et faits marquants et un bilan prospectif qui pourra prendre différentes formes en fonction des groupes (\eg feuille de route, bilan d'un sondage, interviews, ...). 

%Afin de mieux voir les évolutions, nous rappelons quels étaient les groupes de travail sur la période 2016--2020:
%\TODO{Mettre à jour les porteurs}
\silent{
{	\centering
		\begin{tabular}{|p{6cm}p{8.5cm}|}
\hline
\bf Groupes (2016-2019) & \bf Responsables \\
\hline
\hline
Action AFSEC & Claude Jard, IRISA, Université de Nantes\\
\footnotesize Approches Formelles des Systèmes Embarqués Communicants 
             &Olivier H. Roux, IRCCyN, École Centrale de Nantes \\
\hline
COMPILATION
& Florian Brandner, ENSTA\\
& Laure Gonnord, LIP , Université Claude Bernard Lyon 1\\
& Fabrice Rastello, INRIA/LIP, ENS Lyon \\
\hline
ESE 
& Martin Monperrus, CRIStAL, Université de Lille\\
\footnotesize Génie Logiciel Empirique
& Jean-Rémy Falleri, LABRI, ENSEIRB-MATMECA\\
\hline

% chg de repsonsables en cours
%FORWAL & Yohan Boichut, LIFO, Université d'Orléans\\
%\footnotesize  Formalismes et Outils pour la Vérification et la Validation  
%       & Pierre-Cyrille HEAM, LIFC, Université de Franche-Comté \\

GLACE 
& Sébastien Mosser, I3S, Université de Nice\\
\footnotesize Génie Logiciel pour les systèmes Cyber-physiquEs
&Romain Rouvoy, CRIStAL, Université Lille 1\\
\hline
Action IDM
& Eric Cariou, LIUPPA, Université de Pau et des pays de l'Adour\\
\footnotesize Ingénierie Dirigée par les Modèles 
& Clémentine Nébut, LIRMM, Université de Montpellier\\
\hline
IE 
& Régine Laleau, LACL, Université de Paris-Est Créteil\\
\footnotesize Ingénierie des Exigences
& Camille Salinesi, CRI, Université Paris 1 Panthéon-Sorbone\\
\hline

% se regroupe dans LTP
%LaMHa & Gaétan Hains, LACL, Université de Paris-Est\\
%\footnotesize Langages et Modèles de Haut-niveau pour la programmation
%parallèle, distribuée, de grilles de calcul et Applications 
%      & Frédéric Gava, LACL, Université de Paris-Est\\

LTP & Sandrine Blazy, IRISA, Université de Rennes\\
\footnotesize Langages, Types et Preuves 
    & Marc Pouzet,  LIENS, Membre Junior IUF, ENS, Université Pierre et Marie Curie\\
\hline

MFDL 
& Aurélie Hurault, IRIT, ENSEEIHT\\
\footnotesize  Méthodes Formelles
& Akram Idani, LIG, Université Joseph Fourier\\
\footnotesize  dans le Développement Logiciel 
& Virginie Wiels, ONERA\\
\hline

% chg de repsonsables en cours
MTV2 &Nikolai Kosmatov, CEA-LIST\\
\footnotesize Méthodes de test pour la validation et la vérification  
     & Lydie du Bousquet, LIG, Université Grenoble\\
\hline

RIMEL 
& Nicolas Anquetil, CRIStAL, Université de Lille\\
\footnotesize Rétro-Ingénierie, Maintenance et Evolution des Logiciels
& Christelle Urtado, Mines d'Alès\\
\hline

SdS 
& Flavio Oquendo, IRISA, Université de Rennes\\
\footnotesize Systèmes de Systèmes
& Axel Legay, IRISA, Inria\\
& Khalil Drira, LAAS, CNRS\\

\hline 
\textbf{Action transversale}&\\
\hline 

Action ALROB & Jacques Malenfant, LIP6\\
\footnotesize
Architecture logicielles pour la robotique 
             &David Andreu, LIRMM\\
\footnotesize
autonome et les systèmes auto-adaptables
             &Noury Bouraqadi, Mines de Douai\\
             &Serge Stinckwich, UCBN \& UMMISCO\\
\hline

\hline


\hline
		\end{tabular}
}

%A noter qu'un groupe de travail ``Compilation'', animé par Laure Gonnord (LIFL) et Fabrice Rastello (LIP)
%  s'est formé en cours de quadriennal et a organisé des journées de travail, soutenues
%par le GDR GPL, en 2010 et 2011.
\medskip

\noindent
%Par ailleurs plus de la moitié des responsables de groupes ont été renouvelés pour ce quadriennal. 
%Les nouveaux responsables sont Nicolas Anquetil, Sandrine Blazy, Eric Cariou, Jean-Rémy Falleri, Aurélie Hurault, Régine Laleau, Martin Monperrus, Sébastien Mosser, Clémentine Nébut, Camille Salinesi et Christelle Urtado.
%\pem{nommer les nouveaux entrants?}
}


%Tous les groupes de travail ont participé à la sélection des articles présentés lors des trois éditions des journées nationales. Tous les groupes ont organisé des réunions de travail ou ont contribué à l'organisation de conférences nationales.

\subsection{Comité  Scientifique}
Le comité scientifique a être recomposé sous la direction de Yamine Ait Ameur.
Le comité scientifique aura pour missions de susciter de nouvelles actions et de donner son avis sur les actions du GDR, en particulier lors d'appels à actions spécifiques ou d'actions prospectives. 
Il se réunira une fois par an lors des journées nationales avec le comité de direction du GDR GPL et les responsables des groupes de travail et actions transverses pour
faire le bilan de l'année, mais également pour proposer de nouvelles actions.
De nouvelles personnalités, notées avec *, ont intégré le comité scientifique pour la période 2021-2025, soit {\raise.17ex\hbox{$\scriptstyle\sim$}}50\% de renouvellement. 





\begin{itemize}
    \item Franck Barbier (LIUPPA, Pau)
    \item Sandrine Blazy*  (IRISA, Université de Rennes)
    \item Benoit Baudry* (KTH Royal Institute of Technology in Stockholm, Sweden)
    \item Roberto Di Cosmo (PPS, Paris VII)
    \item Laurence Duchien (CRIStAL, Lille)
    \item Marianne Huchard* (LIRMM, Montpellier)
    \item  Stéphane Ducasse (INRIA, Lille)
 %   \item  +++ Marie-Claude Gaudel (LRI, Orsay) 
  %  \item Patrick Farail (IRT Saint Exupéry)x
    \item Jean-Christophe Filliatre*, (LRI, Ile de France)x
    \item Jean-Louis Giavitto (IRCAMS, Paris)
    \item Yann-Gaël Guéhéneuc (Polytech, Montréal)
    \item Olivier Hermant (Mines Paris)
    \item Régine Laleau* (LACL, Université de Paris-Est Créteil)
    \item Pascale Le Gall (MAS, Centrale Paris)
    \item Florence  Maraninchi* (INP, Verimag, Grenoble)
    \item Pierre-Etienne Moreau* (LORIA, Nancy)
%    \item --- Valérie Issarny (INRIA, Rocquencourt)
%    \item -- Jean-Marc Jézéquel (IRISA, Rennes)
   \item Dominique Méry (LORIA, Nancy)
  %  \item Christel Seguin (ONERA, Toulouse)x
    \item Julien Signoles* (CEA LIST)
 \end{itemize}   

 \medskip

 \noindent

\subsection{Outils de communication}

%\subsubsection{Outils électroniques}

Les outils électroniques existants continueront de nous aider dans la communication vers la communauté du GDR :

\begin{itemize}
\item la liste de diffusion \texttt{gdr.gpl@imag.fr}  continuera à fonctionner
  pour l'annonce des événements soutenus ou organisés par le GDR, ainsi que les
  annonces d'emplois liées à la communauté avec les mêmes règles que
  précédemment (liste modérée avec abonnés ayant un droit de postage);

\item le site web  \texttt{http://gdr-gpl.cnrs.fr}  permet la présentation du
  GDR et de ses activités. Nous ferons évoluer le site, éventuellement en connectant d'autres outils, dans les mois qui
  viennent de façon à proposer un contenu plus fourni avec par exemple la liste des sujets de thèse et des soutenances.
  Cette évolution passera par un site ouvert à plus de contributeurs;
 
\item un compte Twitter a été créé en 2013 et continuera d'être alimenté;

\item la mise en place de nouveaux outils sera étudiée afin de construire une cartographie des compétences des laboratoires dans le domaine du Génie de la Programmation et du Logiciel. Nous allons commencer par un sondage qui nous permettra de poser les bases. Puis nous mettrons en place des outils permettant une mise à jour de la carte par les équipes et les groupes de travail.
\end{itemize}

%Nouvelles actions
%Dénombrer les personnes dans les équipes
%Carte de compétences des laboratoires
%Bulletin Lettre du GDR



\section{Aspects administratifs et répartition de la dotation}\label{budget}

\subsection{Aspects administratifs}
%\TODO{
Tous les aspects administratifs liés à la gestion du budget, le paiement des divers versements et
subventions accordées par le GDR GPL, seront effectués sous la responsabilité
de Frédéric Fontaine (Responsable Administratif, I3S).
%}


\subsection{Répartition de la dotation}

La dotation sera, comme dans le précédent quinquennal, répartie entre l'organisation des journées nationales, les réunions des groupes de travail, le soutien à l'école Jeunes Chercheur·e·s. en Programmation et les actions de mobilités. Viendront ensuite le financement d'actions ponctuelles.

\section{Conclusion}\label{conclusion}

Ce  quatrième renouvellement du GDR Génie de la Programmation et du Logiciel aura
pour principal objectif de renforcer la cohésion et la visibilité de notre communauté scientifique. La réussite d'un tel objectif n'est pas mesurable, cependant voici les leviers que nous souhaitons utiliser et qui, dans une certaine mesure, nous permettront de justifier de la réussite de ce quinquennal.

Par des groupes de travail ayant des contours différents en fonction des thématiques abordées nous souhaitons proposer une plus grande variété et complémentarité des actions d'animation. Bien évidemment, nous reconduirons les journées nationales qui sont devenues le rendez-vous phare de la communauté, mais nous les enrichirons par des actions spécifiques (par exemple, des ateliers sur la préparation aux concours, une sensibilisation à la reproductibilité des résultats en GL sous l'impulsion du mouvement Software Heritage). Les journées thématiques organisées par les groupes de travail au sein de notre communauté et de manière transverse à d'autres GDRs ou actions animeront les réflexions et diffuseront nos travaux sous des formes différentes.


Notre communauté a déjà de nombreux échanges avec les IRTs et entreprises. Le GDR aura pour objectif d'au moins cartographier ces relations par secteurs économiques, par nature des contrats\ldots, afin non seulement de les mettre en exergue mais également de mettre en relief nos besoins en matière d'interactions avec le monde de l'entreprise pour conduire nos recherches.
Nous souhaitons également clarifier la richesse de la communauté française scientifique en GL dans toute sa complémentarité en en donnant une vision plus globale diffusable à l'international.


Nous sommes bien conscients de la difficulté à attirer d’excellents jeunes chercheurs en Génie de la Programmation et du Logiciel, ceci dans le cadre actuel très contraint, mais aussi de par l’attractivité des postes à l’international dans notre domaine, que ce soit dans le monde académique ou industriel. Cependant, nous souhaitons faire de notre mieux pour former et rendre attractives nos équipes en 
%Dans le contexte actuel de très forte demande des entreprises et d'une orientation outrancière vers le tout "IA", il semble difficile de prétendre augmenter le nombre de jeunes chercheurs en GL. 
%Cependant, nous souhaitons 
reconduisant et amplifiant les actions en direction des jeunes chercheur·e·s 
%de façon à préparer l'avenir 
dont : les bourses de mobilité, le soutien à l'École des Jeunes
Chercheurs en Programmation, l'accès gratuit aux journées nationales, le prix de
thèse du GDR GPL étendu avec un prix spécial pour les thèses réalisées dans un contexte industriel. 

%Ces actions seront menés par les membres du bureau avec l'aide des groupes de travail.\\

%Finalement, les outils de communication (la liste de diffusion et le site web) seront étendus par une cartographie des compétences à destination de nos partenaires académiques et industriels.
En conclusion, le succès de ce quinquennal passera par la mobilisation des groupes
de travail, du comité de direction, du comité scientifique et de la communauté au sens large. Le travail commencé avec tous ces acteurs est déjà très prometteur. Fort des mandats précédents qui lui ont donné ses racines, nul doute que notre communauté trouvera dans ce nouveau quinquennat de son GDR, un support pour aborder ensemble les défis auxquels le monde du logiciel doit répondre.




\section{Annexe}


\begin{table}
\centering
\caption{TableName}
\begin{tabular}{|l|l|l|}
\hline

Xavier & Blanc & Bordeaux University \\ \hline
Mireille & Blay-Fornarino & Université Nice Sophia Antipolis, I3S \\ \hline
Sandrine & Blazy & University of Rennes 1 - IRISA \\ \hline
Simon & Bliudze & INRIA \\ \hline
Philippe & Collet & Université Côte d'Azur - CNRS/I3S, FR \\ \hline
Evelyne & Contejean & LRI, CNRS, Univ Paris-Sud, Orsay \\ \hline
Steven & Costiou & RMoD team, Inria Lille - Nord Europe \\ \hline
Helene & Coullon & INRIA \\ \hline
Frederic & Dadeau & FEMTO-ST \\ \hline
Pierre-Evariste & Dagand & LIP6 - CNRS \\ \hline
Thomas & Degueule & Univ. Bordeaux, CNRS, Bordeaux INP, LaBRI, UMR 5800, F-33400, Talence, France \\ \hline
Lydie & Du Bousquet & LIG \\ \hline
Catherine & Dubois & ENSIIE-Samovar \\ \hline
Stéphane & Ducasse & INRIA \\ \hline
Laurence & Duchien & University of Lille \\ \hline
Ernesto & Exposito & Université de Pau et des Pays de l'Adour \\ \hline
Laure & Gonnord & University of Lyon / Laboratoire d'Informatique du Parallélisme \\ \hline
Arnaud & Gotlieb & SIMULA Research Laboratory \\ \hline
Yann-Gaël & Guéhéneuc & Concordia University \\ \hline
Marianne & Huchard & LIRMM, Université de Montpellier et CNRS \\ \hline
Jean-Marc & Jézéquel & University of Rennes \\ \hline
Djamel Eddine & Khelladi & DIVERSE Team, IRISA-INRIA, CNRS, Université Rennes 1 \\ \hline
Jacques & Klein & University of Luxembourg \\ \hline
Regine & Laleau & Paris Est Creteil University \\ \hline
Daniel & Le Berre & CNRS - Université d'Artois \\ \hline
Pascale & Le Gall & CentraleSupelec \\ \hline
Yves & Ledru & Laboratoire d'Informatique de Grenoble - Université Grenoble Alpes \\ \hline
David & Monniaux & CNRS / VERIMAG \\ \hline
Pierre-Etienne & Moreau & INRIA-LORIA Nancy \\ \hline
Thomas & Polacsek & ONERA \\ \hline
Romain & Rouvoy & Univ. Lille / Inria / IUF \\ \hline
Helene & Waeselynck & LAAS-CNRS \\ \hline
Tewfik & Ziadi & Sorbonne Université-CNRS 7606, LIP6 \\ \hline

\end{tabular}
\end{table}
\end{document}
